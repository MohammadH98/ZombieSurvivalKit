\documentclass{article}

\usepackage{booktabs}
\usepackage{tabularx}

\title{SE 3XA3: Development Plan\\Zombie Survival}

\author{Team 6
		\\ Mohammad Hussain, hussam17
		\\ Brian Jonatan,  jonatans
		\\ Shivaansh Prasann, prasanns
}

\date{September 26, 2018}

%\input{../Comments}

\begin{document}

\begin{table}[hp]
\caption{Revision History} \label{TblRevisionHistory}
\begin{tabularx}{\textwidth}{llX}
\toprule
\textbf{Date} & \textbf{Developer(s)} & \textbf{Change}\\
\midrule
2018-06-26 & Muhammad, Brian, Shivaansh & Added first change for every section\\
Date2 & Name(s) & Description of changes\\
... & ... & ...\\
\bottomrule
\end{tabularx}
\end{table}

\newpage

\maketitle

Put your introductory blurb here.

\section{Team Meeting Plan}
Mondays: 2:30PM - 4:30PM (HSL Library)\\
Tuesdays: 5:00 PM - 7:00 PM (HSL Library) (as and when needed)\\
Tuesdays: 7:00 PM - 9:00 PM (ITB 236 Lab)\\
Wednesdays: 12:30 PM - 2:30 PM (ITB 236 Lab)

\section{Team Communication Plan}
Discord Server (Audio calls for meetings)\\
Facebook Group Chat for general inquiries

\section{Team Member Roles}
Brian Jonatan - Developer and Tester\\
Mohammad Hussain - Developer and Scribe\\
Shivaansh Prasann - Developer and Project Manager

\section{Git Workflow Plan}
Each developer has their own branch. Code reviews will be conducted during meetings and once all individual branches are fully functional each branch will be merged to master. Tags will be used after pushing each deliverable for the project.

\section{Proof of Concept Demonstration Plan}
The main challenges with this project would be the following: \newline
\newline
1) Testing: For testing purposes, we need to perform rigorous playtesting, which incorporates playing a certain level again and again to uncover any performance bugs, as well as two-way testing, which tests one component of the system based on the assumption that another component of the system (on which the current component is dependent on) is correct.
\newline
\newline
2) Minimap: Since no one on the team has previous experience on creating a minimap for a game, this is bound to be a challenging aspect of this project and therefore should be demonstrated in the Proof of Concept.
\newline
\newline
3) Movement on a plane: Moving the player around is the most elementary component of our project, hence this will be included in the Proof of Concept.

\section{Technology}
Programming Language: C\# \\
IDE: Visual Studio\\
Testing Framework: Unity3D\\
Document Generation: XML Documentation Comments

\section{Coding Style}
C\# Coding Conventions (C\# Programming Guide)\\
URL: https://docs.microsoft.com/en-us/dotnet/csharp/programming-guide/inside-a-program/coding-conventions

\section{Project Schedule}

Provide a pointer to your Gantt Chart.

\section{Project Review}

\end{document}